% Section 1-1
\section{电路和电路模型}

\subsection{电路}
由电阻器、电容器、线圈、变压器、晶体管、运算放大器、传输线、电池、发电机和信号发生器等
电器器件和设备连接而成的电路,称为实际电路。
根据实际电路的集合尺寸($d$)与其工作信号波长($\lambda$)的关系,可以将它们分为两大类:
满足 $d\ll\lambda$ 条件的电路称为集总参数电路,
其特点是电路中任意两个端点间的电压和流入任一器件端钮的电流是完全确定的,
与器件的几何尺寸和空间位置无关。不满足 $d\ll\lambda$ 条件的另一类电路称为分布参数电路。
本书只讨论集总参数电路,今后简称为电路。

\subsection{电路模型}
